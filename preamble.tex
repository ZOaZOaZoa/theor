\documentclass[a4paper, 14pt,russian]{extarticle}

\usepackage[russian]{babel}
\usepackage[T2A]{fontenc}
\usepackage[utf8]{inputenc}
%Соответствующий математический шрифт для Times new roman
\usepackage{newtxmath}
\usepackage{fontspec} 
%Times new roman
\defaultfontfeatures{Ligatures={TeX},Renderer=Basic} 
\setmainfont[Ligatures={TeX,Historic}]{Times New Roman}

%Геометрия
\usepackage{geometry}
\geometry{top=20mm}
\geometry{bottom=15mm}
\geometry{left=20mm}
\geometry{right=15mm}
\usepackage{setspace}
%Нормальные дроби через запятую
\usepackage{ncccomma}

\newcommand{\changefont}{%
	\fontsize{12}{11}\selectfont
}
\newcommand{\normE}[1]{
	\lvert\lvert {#1} \rvert\rvert_2
}

%Заголовки
\usepackage{fancyhdr}
%\pagestyle{fancy}
%\fancyhf{}
%\renewcommand{\sectionmark}[1]{\markright{#1}}
%\fancyhead[R]{\changefont \slshape \leftmark}
%\fancyhead[L]{\changefont \slshape \rightmark}
%\newcommand{\ssubsection}[1]{\subsection*{#1}
%	\addcontentsline{toc}{subsection}{#1}
%	\markright{#1}{}}
\cfoot{\thepage}

%\полуторный интервал
\setstretch{1.15}
\setlength{\parindent}{1.25cm}

\usepackage{amsmath, amsfonts, mathtools}
\usepackage{physics}
\usepackage{indentfirst}
\usepackage{xcolor}
\usepackage{alltt}
\usepackage{graphicx}
\usepackage{wrapfig}
%Настройка ссылок
\usepackage{hyperref}
%\usepackage{upgreek}
%\renewcommand{\beta}{\upbeta}
\hypersetup{
	colorlinks,
	citecolor=black,
	filecolor=black,
	linkcolor=black,
	urlcolor=black
}
\usepackage{caption}
\DeclareCaptionLabelSeparator{dot}{. }
\captionsetup{justification=centering,labelsep=dot}
\usepackage{titlesec}

%Формат заголовков
\titleformat{\section}{\bfseries\filcenter\Large}{\thesection}{1em}{}
\titleformat{\subsection}{\bfseries\filcenter\large}{\thesubsection}{1em}{}
\titleformat{\subsubsection}{\bfseries\filcenter\normalsize}{\thesubsubsection}{1em}{}

\usepackage{chngcntr}

%Включить в нумерацию картинок раздел
\counterwithin{figure}{section}

%Листинги кода и их стили
\usepackage{listings}
\lstdefinestyle{c++} {
	language=C++,
	breaklines=true,
	frame=single,
	numbers=left,
	basicstyle=\footnotesize\ttfamily,
	keywordstyle=\bfseries\color{green!40!black},
	commentstyle=\itshape\color{purple!40!black},
	identifierstyle=\color{blue},
	backgroundcolor=\color{gray!10!white},
}

\lstdefinestyle{python}{
	language=Python,
	breaklines=true,
	frame=single,
	numbers=left,
	basicstyle=\footnotesize\ttfamily,
	keywordstyle=\bfseries\color{green!40!black},
	frame=lines
	basicstyle=\footnotesize
}

\lstdefinestyle{cmd}{
	breaklines=true,
	frame=single,
	basicstyle=\footnotesize\ttfamily,
	frame=lines
	basicstyle=\footnotesize
}